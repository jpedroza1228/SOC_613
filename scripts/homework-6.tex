\documentclass[]{article}
\usepackage{lmodern}
\usepackage{amssymb,amsmath}
\usepackage{ifxetex,ifluatex}
\usepackage{fixltx2e} % provides \textsubscript
\ifnum 0\ifxetex 1\fi\ifluatex 1\fi=0 % if pdftex
  \usepackage[T1]{fontenc}
  \usepackage[utf8]{inputenc}
\else % if luatex or xelatex
  \ifxetex
    \usepackage{mathspec}
  \else
    \usepackage{fontspec}
  \fi
  \defaultfontfeatures{Ligatures=TeX,Scale=MatchLowercase}
\fi
% use upquote if available, for straight quotes in verbatim environments
\IfFileExists{upquote.sty}{\usepackage{upquote}}{}
% use microtype if available
\IfFileExists{microtype.sty}{%
\usepackage{microtype}
\UseMicrotypeSet[protrusion]{basicmath} % disable protrusion for tt fonts
}{}
\usepackage[margin=1in]{geometry}
\usepackage{hyperref}
\hypersetup{unicode=true,
            pdftitle={Homework 6},
            pdfauthor={JP},
            pdfborder={0 0 0},
            breaklinks=true}
\urlstyle{same}  % don't use monospace font for urls
\usepackage{color}
\usepackage{fancyvrb}
\newcommand{\VerbBar}{|}
\newcommand{\VERB}{\Verb[commandchars=\\\{\}]}
\DefineVerbatimEnvironment{Highlighting}{Verbatim}{commandchars=\\\{\}}
% Add ',fontsize=\small' for more characters per line
\usepackage{framed}
\definecolor{shadecolor}{RGB}{248,248,248}
\newenvironment{Shaded}{\begin{snugshade}}{\end{snugshade}}
\newcommand{\AlertTok}[1]{\textcolor[rgb]{0.94,0.16,0.16}{#1}}
\newcommand{\AnnotationTok}[1]{\textcolor[rgb]{0.56,0.35,0.01}{\textbf{\textit{#1}}}}
\newcommand{\AttributeTok}[1]{\textcolor[rgb]{0.77,0.63,0.00}{#1}}
\newcommand{\BaseNTok}[1]{\textcolor[rgb]{0.00,0.00,0.81}{#1}}
\newcommand{\BuiltInTok}[1]{#1}
\newcommand{\CharTok}[1]{\textcolor[rgb]{0.31,0.60,0.02}{#1}}
\newcommand{\CommentTok}[1]{\textcolor[rgb]{0.56,0.35,0.01}{\textit{#1}}}
\newcommand{\CommentVarTok}[1]{\textcolor[rgb]{0.56,0.35,0.01}{\textbf{\textit{#1}}}}
\newcommand{\ConstantTok}[1]{\textcolor[rgb]{0.00,0.00,0.00}{#1}}
\newcommand{\ControlFlowTok}[1]{\textcolor[rgb]{0.13,0.29,0.53}{\textbf{#1}}}
\newcommand{\DataTypeTok}[1]{\textcolor[rgb]{0.13,0.29,0.53}{#1}}
\newcommand{\DecValTok}[1]{\textcolor[rgb]{0.00,0.00,0.81}{#1}}
\newcommand{\DocumentationTok}[1]{\textcolor[rgb]{0.56,0.35,0.01}{\textbf{\textit{#1}}}}
\newcommand{\ErrorTok}[1]{\textcolor[rgb]{0.64,0.00,0.00}{\textbf{#1}}}
\newcommand{\ExtensionTok}[1]{#1}
\newcommand{\FloatTok}[1]{\textcolor[rgb]{0.00,0.00,0.81}{#1}}
\newcommand{\FunctionTok}[1]{\textcolor[rgb]{0.00,0.00,0.00}{#1}}
\newcommand{\ImportTok}[1]{#1}
\newcommand{\InformationTok}[1]{\textcolor[rgb]{0.56,0.35,0.01}{\textbf{\textit{#1}}}}
\newcommand{\KeywordTok}[1]{\textcolor[rgb]{0.13,0.29,0.53}{\textbf{#1}}}
\newcommand{\NormalTok}[1]{#1}
\newcommand{\OperatorTok}[1]{\textcolor[rgb]{0.81,0.36,0.00}{\textbf{#1}}}
\newcommand{\OtherTok}[1]{\textcolor[rgb]{0.56,0.35,0.01}{#1}}
\newcommand{\PreprocessorTok}[1]{\textcolor[rgb]{0.56,0.35,0.01}{\textit{#1}}}
\newcommand{\RegionMarkerTok}[1]{#1}
\newcommand{\SpecialCharTok}[1]{\textcolor[rgb]{0.00,0.00,0.00}{#1}}
\newcommand{\SpecialStringTok}[1]{\textcolor[rgb]{0.31,0.60,0.02}{#1}}
\newcommand{\StringTok}[1]{\textcolor[rgb]{0.31,0.60,0.02}{#1}}
\newcommand{\VariableTok}[1]{\textcolor[rgb]{0.00,0.00,0.00}{#1}}
\newcommand{\VerbatimStringTok}[1]{\textcolor[rgb]{0.31,0.60,0.02}{#1}}
\newcommand{\WarningTok}[1]{\textcolor[rgb]{0.56,0.35,0.01}{\textbf{\textit{#1}}}}
\usepackage{graphicx,grffile}
\makeatletter
\def\maxwidth{\ifdim\Gin@nat@width>\linewidth\linewidth\else\Gin@nat@width\fi}
\def\maxheight{\ifdim\Gin@nat@height>\textheight\textheight\else\Gin@nat@height\fi}
\makeatother
% Scale images if necessary, so that they will not overflow the page
% margins by default, and it is still possible to overwrite the defaults
% using explicit options in \includegraphics[width, height, ...]{}
\setkeys{Gin}{width=\maxwidth,height=\maxheight,keepaspectratio}
\IfFileExists{parskip.sty}{%
\usepackage{parskip}
}{% else
\setlength{\parindent}{0pt}
\setlength{\parskip}{6pt plus 2pt minus 1pt}
}
\setlength{\emergencystretch}{3em}  % prevent overfull lines
\providecommand{\tightlist}{%
  \setlength{\itemsep}{0pt}\setlength{\parskip}{0pt}}
\setcounter{secnumdepth}{0}
% Redefines (sub)paragraphs to behave more like sections
\ifx\paragraph\undefined\else
\let\oldparagraph\paragraph
\renewcommand{\paragraph}[1]{\oldparagraph{#1}\mbox{}}
\fi
\ifx\subparagraph\undefined\else
\let\oldsubparagraph\subparagraph
\renewcommand{\subparagraph}[1]{\oldsubparagraph{#1}\mbox{}}
\fi

%%% Use protect on footnotes to avoid problems with footnotes in titles
\let\rmarkdownfootnote\footnote%
\def\footnote{\protect\rmarkdownfootnote}

%%% Change title format to be more compact
\usepackage{titling}

% Create subtitle command for use in maketitle
\providecommand{\subtitle}[1]{
  \posttitle{
    \begin{center}\large#1\end{center}
    }
}

\setlength{\droptitle}{-2em}

  \title{Homework 6}
    \pretitle{\vspace{\droptitle}\centering\huge}
  \posttitle{\par}
    \author{JP}
    \preauthor{\centering\large\emph}
  \postauthor{\par}
      \predate{\centering\large\emph}
  \postdate{\par}
    \date{2/14/2020}


\begin{document}
\maketitle

Homework Assignment \#6 In this assignment you will again be working
with a subset of the publicly available version of the data from the
National Longitudinal Study of Adolescent to Adult Health (Add Health).
This data can be found on the course website in Canvas under the
``Files'' tab. Look for data file ``AddHealth.dta.'' Use the STATA
commands handout (also available in the Assignments tab) to help you
complete this assignment. On the due date please hand in a hardcopy of
your STATA output and answers to all questions in this assignment. You
can either hand in a document with your answers (e.g., Word doc) with
STATA output attached or you can incorporate answers to the questions
directly into the STATA output in the form of comments. I prefer typed
assignments in Times New Roman size 12 or Arial size 11. The following
is a basic description of variables you will be using in this
assignment. § aid = a unique id number assigned to each adolescent
respondent § schoolid = a unique id number assigned to each school in
the data set § smoke\_30days\_w1 = a (semi) continuous measure of the
number of days the student smoked cigarettes in the past 30 days § cesd
= a continuous measure of depression (score on the CESD survey) § sex =
adolescent sex (it is unclear whether this survey item was more closely
measuring sex or gender. For our purposes assume it measures gender.) o
1 = male o 2 = female § parent\_highestedu = highest educational
attainment of either parent/guardian/parent-figure of adolescents in the
sample: o 1 = Less than high school (no HS degree) o 2 = Completed high
school or equivalent o 3 = Some college (no degree) o 4 = College degree
or more § age\_w1 = age in years (continuous measure) reported at wave 1

\begin{Shaded}
\begin{Highlighting}[]
\KeywordTok{library}\NormalTok{(tidyverse)}
\end{Highlighting}
\end{Shaded}

\begin{verbatim}
## -- Attaching packages -------------------------------------------------------------------------------------------------------------------------------------------- tidyverse 1.2.1 --
\end{verbatim}

\begin{verbatim}
## v ggplot2 3.2.0     v purrr   0.3.2
## v tibble  2.1.3     v dplyr   0.8.3
## v tidyr   1.0.2     v stringr 1.4.0
## v readr   1.3.1     v forcats 0.4.0
\end{verbatim}

\begin{verbatim}
## -- Conflicts ----------------------------------------------------------------------------------------------------------------------------------------------- tidyverse_conflicts() --
## x dplyr::filter() masks stats::filter()
## x dplyr::lag()    masks stats::lag()
\end{verbatim}

\begin{Shaded}
\begin{Highlighting}[]
\KeywordTok{library}\NormalTok{(readstata13)}
\end{Highlighting}
\end{Shaded}

\begin{verbatim}
## Warning: package 'readstata13' was built under R version 3.6.2
\end{verbatim}

\begin{Shaded}
\begin{Highlighting}[]
\KeywordTok{library}\NormalTok{(lme4)}
\end{Highlighting}
\end{Shaded}

\begin{verbatim}
## Loading required package: Matrix
\end{verbatim}

\begin{verbatim}
## 
## Attaching package: 'Matrix'
\end{verbatim}

\begin{verbatim}
## The following objects are masked from 'package:tidyr':
## 
##     expand, pack, unpack
\end{verbatim}

\begin{Shaded}
\begin{Highlighting}[]
\KeywordTok{library}\NormalTok{(psych)}
\end{Highlighting}
\end{Shaded}

\begin{verbatim}
## 
## Attaching package: 'psych'
\end{verbatim}

\begin{verbatim}
## The following objects are masked from 'package:ggplot2':
## 
##     %+%, alpha
\end{verbatim}

\begin{Shaded}
\begin{Highlighting}[]
\KeywordTok{library}\NormalTok{(optimx)}
\end{Highlighting}
\end{Shaded}

\begin{verbatim}
## Warning: package 'optimx' was built under R version 3.6.2
\end{verbatim}

\begin{Shaded}
\begin{Highlighting}[]
\KeywordTok{library}\NormalTok{(lmerTest)}
\end{Highlighting}
\end{Shaded}

\begin{verbatim}
## Warning: package 'lmerTest' was built under R version 3.6.2
\end{verbatim}

\begin{verbatim}
## 
## Attaching package: 'lmerTest'
\end{verbatim}

\begin{verbatim}
## The following object is masked from 'package:lme4':
## 
##     lmer
\end{verbatim}

\begin{verbatim}
## The following object is masked from 'package:stats':
## 
##     step
\end{verbatim}

\begin{Shaded}
\begin{Highlighting}[]
\KeywordTok{library}\NormalTok{(dfoptim)}

\KeywordTok{options}\NormalTok{(}\DataTypeTok{max.print =} \DecValTok{99999}\NormalTok{)}
\KeywordTok{options}\NormalTok{(}\DataTypeTok{scipen =} \DecValTok{999}\NormalTok{)}

\KeywordTok{getwd}\NormalTok{()}
\end{Highlighting}
\end{Shaded}

\begin{verbatim}
## [1] "E:/UO/R Projects/SOC 613/scripts"
\end{verbatim}

\begin{Shaded}
\begin{Highlighting}[]
\KeywordTok{set.seed}\NormalTok{(}\DecValTok{232020}\NormalTok{)}

\NormalTok{data <-}\StringTok{ }\KeywordTok{read.dta13}\NormalTok{(}\StringTok{"E:/UO/R Projects/SOC 613/data/AddHealth.dta"}\NormalTok{)}

\NormalTok{data <-}\StringTok{ }\NormalTok{data }\OperatorTok\StringTok{ }
\StringTok{  }\NormalTok{dplyr}\OperatorTok{::}\KeywordTok{select}\NormalTok{(aid, }
\NormalTok{                schoolid, }
\NormalTok{                smoke_30days_w1,}
\NormalTok{                cesd,}
\NormalTok{                sex,}
\NormalTok{                parent_highestedu,}
\NormalTok{                age_w1) }\OperatorTok\StringTok{ }
\StringTok{  }\KeywordTok{rename}\NormalTok{(}\DataTypeTok{student =}\NormalTok{ aid,}
         \DataTypeTok{school =}\NormalTok{ schoolid,}
         \DataTypeTok{smoke =}\NormalTok{ smoke_30days_w1,}
         \DataTypeTok{depression =}\NormalTok{ cesd,}
         \DataTypeTok{parent_ed =}\NormalTok{ parent_highestedu,}
         \DataTypeTok{age =}\NormalTok{ age_w1)}

\KeywordTok{colnames}\NormalTok{(data)}
\end{Highlighting}
\end{Shaded}

\begin{verbatim}
## [1] "student"    "school"     "smoke"      "depression" "sex"       
## [6] "parent_ed"  "age"
\end{verbatim}

Tasks: 1) Briefly explain the difference between ``binary'' and
``binomial'' outcomes.

Answer: Binary outcomes are when data are 0's and 1's and the outcome is
the probability of having something (yes) for an individual. Binomial
outcomes are cells of proporitions in a group of people.

\begin{enumerate}
\def\labelenumi{\arabic{enumi})}
\setcounter{enumi}{1}
\tightlist
\item
  Briefly explain the difference between ``Odds'' and ``Odds Ratio.''
\end{enumerate}

Answer: Odds are the values of probabilities of yes divided by the
probabilities of no. Odds ratios are the odds of one group divided by
the odds of the other group to make comparisons.

\begin{enumerate}
\def\labelenumi{\arabic{enumi})}
\setcounter{enumi}{2}
\tightlist
\item
  Smoking:
\end{enumerate}

\begin{enumerate}
\def\labelenumi{\alph{enumi}.}
\tightlist
\item
  Generate a variable called ``smoker'' which = 1 if the student smoked
  on 1 or more days in the past 30 during wave 1, and = 0 if not.
\end{enumerate}

\begin{Shaded}
\begin{Highlighting}[]
\KeywordTok{describe}\NormalTok{(data}\OperatorTok{$}\NormalTok{smoke, }\DataTypeTok{na.rm =} \OtherTok{TRUE}\NormalTok{)}
\end{Highlighting}
\end{Shaded}

\begin{verbatim}
##    vars    n mean    sd median trimmed mad min max range skew kurtosis
## X1    1 5591 4.92 10.08      0     2.4   0   0  30    30 1.85      1.7
##      se
## X1 0.13
\end{verbatim}

\begin{Shaded}
\begin{Highlighting}[]
\NormalTok{data }\OperatorTok\StringTok{ }
\KeywordTok{ggplot}\NormalTok{(}\KeywordTok{aes}\NormalTok{(smoke)) }\OperatorTok{+}
\StringTok{  }\KeywordTok{geom_histogram}\NormalTok{(}\DataTypeTok{color =} \StringTok{'white'}\NormalTok{, }\DataTypeTok{fill =} \StringTok{'darkgreen'}\NormalTok{)}
\end{Highlighting}
\end{Shaded}

\begin{verbatim}
## `stat_bin()` using `bins = 30`. Pick better value with `binwidth`.
\end{verbatim}

\begin{verbatim}
## Warning: Removed 913 rows containing non-finite values (stat_bin).
\end{verbatim}

\includegraphics{homework-6_files/figure-latex/unnamed-chunk-2-1.pdf}

\begin{Shaded}
\begin{Highlighting}[]
\CommentTok{# Answer: 3A}
\NormalTok{data <-}\StringTok{ }\NormalTok{data }\OperatorTok\StringTok{ }
\StringTok{  }\KeywordTok{mutate}\NormalTok{(}\DataTypeTok{smoker =} \KeywordTok{case_when}\NormalTok{(smoke }\OperatorTok{<=}\StringTok{ }\DecValTok{0} \OperatorTok{~}\StringTok{ }\DecValTok{0}\NormalTok{,}
\NormalTok{                          smoke }\OperatorTok{>}\StringTok{ }\DecValTok{0} \OperatorTok{~}\StringTok{ }\DecValTok{1}\NormalTok{))}

\NormalTok{data }\OperatorTok\StringTok{ }
\KeywordTok{ggplot}\NormalTok{(}\KeywordTok{aes}\NormalTok{(smoker)) }\OperatorTok{+}
\StringTok{  }\KeywordTok{geom_bar}\NormalTok{(}\DataTypeTok{color =} \StringTok{'white'}\NormalTok{, }\DataTypeTok{fill =} \StringTok{'dodgerblue'}\NormalTok{) }\OperatorTok{+}
\StringTok{  }\KeywordTok{theme_minimal}\NormalTok{()}
\end{Highlighting}
\end{Shaded}

\begin{verbatim}
## Warning: Removed 913 rows containing non-finite values (stat_count).
\end{verbatim}

\includegraphics{homework-6_files/figure-latex/unnamed-chunk-2-2.pdf}

\begin{Shaded}
\begin{Highlighting}[]
\CommentTok{# Answer 3B}
\NormalTok{data }\OperatorTok\StringTok{ }
\StringTok{  }\KeywordTok{drop_na}\NormalTok{(smoker) }\OperatorTok\StringTok{ }
\StringTok{  }\KeywordTok{group_by}\NormalTok{(smoker) }\OperatorTok\StringTok{ }
\StringTok{  }\KeywordTok{summarize}\NormalTok{(}\DataTypeTok{n =} \KeywordTok{n}\NormalTok{()) }\OperatorTok\StringTok{ }
\StringTok{  }\KeywordTok{mutate}\NormalTok{(}\DataTypeTok{freq =}\NormalTok{ n}\OperatorTok{/}\KeywordTok{sum}\NormalTok{(n))}
\end{Highlighting}
\end{Shaded}

\begin{verbatim}
## # A tibble: 2 x 3
##   smoker     n  freq
##    <dbl> <int> <dbl>
## 1      0  3944 0.705
## 2      1  1647 0.295
\end{verbatim}

\begin{enumerate}
\def\labelenumi{\alph{enumi}.}
\setcounter{enumi}{1}
\tightlist
\item
  Obtain statistics on ``smoker'' -- what proportion of respondents are
  past-30 day smokers in wave 1?
\end{enumerate}

Answer: .29 or 29\% of respondents are past smokers.

\begin{enumerate}
\def\labelenumi{\alph{enumi}.}
\setcounter{enumi}{2}
\tightlist
\item
  Is ``smoker'' a binary or binomial outcome?
\end{enumerate}

Answer: Smoker is a binary outcome.

\begin{enumerate}
\def\labelenumi{\arabic{enumi})}
\setcounter{enumi}{3}
\tightlist
\item
  Model 1: Write a logistic Random Intercepts model, where adolescents
  (level 1) are nested in schools (level 2). The outcome of interest is
  ``smoker.'' Include the following FE predictors: ``female'' and
  ``parent education'' (reference level = college degree or more). When
  writing the model be careful to include all steps (micro, macro,
  combined) and with the final combined model be sure to specify whether
  the model is Bernoulli or binomial, and what the variance at each
  level is.
\end{enumerate}

Micro:
\[ logit(\pi_{ij}) = \beta_{0j} + \beta_{1j}(female_{ij}) + \beta_{2j}(lessthanhs_{ij}) + \beta_{3j}(hsgrad_{ij}) + \beta_{4j}(somecollege_{ij}) \]

Macro 1:(Intercept) \[ \beta_{0j} = \beta_0 + \mu_{0j} \]

\[ logit(\pi_{ij}) = \beta_0 + \beta_1(female_{ij}) + \beta_2(lessthanhs_{ij}) + \beta_3(hsgrad_{ij}) + \beta_4(somecollege_{ij}) + \mu_{0j}\]

\[ Smoker_{ij} \sim Bernoulli(\pi_{ij})\]

\[ Level 2: \mu_{0j} \sim N(0, \sigma^2_{u0}) \]

\[  Level 1: Var(Smoker_{ij} | \pi_{ij}) = \pi_{ij}(1 - \pi_{ij})  \]

\begin{enumerate}
\def\labelenumi{\arabic{enumi})}
\setcounter{enumi}{4}
\tightlist
\item
  Using STATA, fit Model 1 and request the ``betas'' be provided in the
  form of Odds and Odds Ratios. Then answer the following questions
  using the results:
\end{enumerate}

\begin{Shaded}
\begin{Highlighting}[]
\NormalTok{data <-}\StringTok{ }\NormalTok{data }\OperatorTok\StringTok{ }
\StringTok{  }\KeywordTok{mutate}\NormalTok{(}\DataTypeTok{parent_ed =} \KeywordTok{recode}\NormalTok{(parent_ed, }\StringTok{'1'}\NormalTok{ =}\StringTok{ 'less_than_hs'}\NormalTok{,}
                                    \StringTok{'2'}\NormalTok{ =}\StringTok{ 'hs_grad'}\NormalTok{,}
                                    \StringTok{'3'}\NormalTok{ =}\StringTok{ 'some_college'}\NormalTok{,}
                                    \StringTok{'4'}\NormalTok{ =}\StringTok{ 'college_degree'}\NormalTok{),}
         \DataTypeTok{sex =} \KeywordTok{recode}\NormalTok{(sex, }\StringTok{'1'}\NormalTok{ =}\StringTok{ 'male'}\NormalTok{,}
                      \StringTok{'2'}\NormalTok{ =}\StringTok{ 'female'}\NormalTok{)) }\OperatorTok\StringTok{ }
\StringTok{  }\KeywordTok{mutate}\NormalTok{(}\DataTypeTok{parent_ed =} \KeywordTok{relevel}\NormalTok{(}\KeywordTok{as.factor}\NormalTok{(parent_ed), }\DataTypeTok{ref =} \StringTok{'college_degree'}\NormalTok{),}
         \DataTypeTok{sex =} \KeywordTok{relevel}\NormalTok{(}\KeywordTok{as.factor}\NormalTok{(sex), }\DataTypeTok{ref =} \StringTok{'male'}\NormalTok{))}

\NormalTok{model1 <-}\StringTok{ }\KeywordTok{glmer}\NormalTok{(smoker }\OperatorTok{~}\StringTok{ }\NormalTok{sex }\OperatorTok{+}\StringTok{ }\NormalTok{parent_ed }\OperatorTok{+}\StringTok{ }\NormalTok{(}\DecValTok{1} \OperatorTok{|}\NormalTok{school), }
                \DataTypeTok{data =}\NormalTok{ data, }
                \DataTypeTok{family =} \KeywordTok{binomial}\NormalTok{(}\DataTypeTok{link =} \StringTok{"logit"}\NormalTok{))}
\KeywordTok{summary}\NormalTok{(model1)}
\end{Highlighting}
\end{Shaded}

\begin{verbatim}
## Generalized linear mixed model fit by maximum likelihood (Laplace
##   Approximation) [glmerMod]
##  Family: binomial  ( logit )
## Formula: smoker ~ sex + parent_ed + (1 | school)
##    Data: data
## 
##      AIC      BIC   logLik deviance df.resid 
##   6464.6   6504.3  -3226.3   6452.6     5507 
## 
## Scaled residuals: 
##     Min      1Q  Median      3Q     Max 
## -1.2067 -0.6784 -0.5058  1.1205  2.9877 
## 
## Random effects:
##  Groups Name        Variance Std.Dev.
##  school (Intercept) 0.3739   0.6115  
## Number of obs: 5513, groups:  school, 132
## 
## Fixed effects:
##                       Estimate Std. Error z value             Pr(>|z|)    
## (Intercept)           -1.28822    0.08733 -14.750 < 0.0000000000000002 ***
## sexfemale             -0.08228    0.06167  -1.334              0.18212    
## parent_edhs_grad       0.49409    0.08407   5.877        0.00000000418 ***
## parent_edless_than_hs  0.38160    0.11974   3.187              0.00144 ** 
## parent_edsome_college  0.40689    0.08140   4.999        0.00000057771 ***
## ---
## Signif. codes:  0 '***' 0.001 '**' 0.01 '*' 0.05 '.' 0.1 ' ' 1
## 
## Correlation of Fixed Effects:
##             (Intr) sexfml prnt_dh_ prn___
## sexfemale   -0.365                       
## prnt_dhs_gr -0.474  0.004                
## prnt_dlss__ -0.361  0.010  0.383         
## prnt_dsm_cl -0.480  0.022  0.510    0.362
\end{verbatim}

\begin{Shaded}
\begin{Highlighting}[]
\CommentTok{# cc_model1 <- confint(model1, parm="beta_") }
\NormalTok{cc_model1_wald <-}\StringTok{ }\KeywordTok{confint}\NormalTok{(model1, }\DataTypeTok{parm =} \StringTok{"beta_"}\NormalTok{, }\DataTypeTok{method =} \StringTok{"Wald"}\NormalTok{)}
\CommentTok{# ctab_model1 <- cbind(est=fixef(model1), cc_model1)}
\NormalTok{ctab_model1_wald <-}\StringTok{ }\KeywordTok{cbind}\NormalTok{(}\DataTypeTok{est =} \KeywordTok{fixef}\NormalTok{(model1), cc_model1_wald)}

\NormalTok{ctab_model1_wald}
\end{Highlighting}
\end{Shaded}

\begin{verbatim}
##                               est      2.5 %      97.5 %
## (Intercept)           -1.28822047 -1.4593938 -1.11704709
## sexfemale             -0.08228076 -0.2031478  0.03858623
## parent_edhs_grad       0.49408711  0.3293065  0.65886776
## parent_edless_than_hs  0.38160468  0.1469090  0.61630037
## parent_edsome_college  0.40689191  0.2473461  0.56643769
\end{verbatim}

\begin{Shaded}
\begin{Highlighting}[]
\NormalTok{rtab_model1_wald <-}\StringTok{ }\KeywordTok{exp}\NormalTok{(ctab_model1_wald)}

\CommentTok{# Answer: Odds & Odds Ratios}
\KeywordTok{print}\NormalTok{(rtab_model1_wald, }\DataTypeTok{digits =} \DecValTok{3}\NormalTok{)}
\end{Highlighting}
\end{Shaded}

\begin{verbatim}
##                         est 2.5 % 97.5 %
## (Intercept)           0.276 0.232  0.327
## sexfemale             0.921 0.816  1.039
## parent_edhs_grad      1.639 1.390  1.933
## parent_edless_than_hs 1.465 1.158  1.852
## parent_edsome_college 1.502 1.281  1.762
\end{verbatim}

\begin{enumerate}
\def\labelenumi{\alph{enumi}.}
\tightlist
\item
  What is the Odds being a smoker among males with a parent who has
  completed a college degree or more?
\end{enumerate}

Answer: The odds of being a smoker that is a male with a parent with a
college degree or more is .28

\begin{enumerate}
\def\labelenumi{\alph{enumi}.}
\setcounter{enumi}{1}
\tightlist
\item
  How might we interpret the Odds Ratio associated with the lowest
  education level (less than high school degree)?
\end{enumerate}

Answer: Adolescents with parents that have less than a high school
degree had 1.47 times the odds of reporting having smoked in the last 30
days than adolescents with parents that have a parent with a college
degree or more.

\begin{enumerate}
\def\labelenumi{\alph{enumi}.}
\setcounter{enumi}{2}
\tightlist
\item
  Regardless of whether or not it is statistically significant, how
  would we interpret the Odds Ratio for ``female'' in this model?
\end{enumerate}

Answer: Females had .92 times the odds of reporting having smoked in the
last 30 days compared to male adolescents.

\begin{enumerate}
\def\labelenumi{\alph{enumi}.}
\setcounter{enumi}{3}
\tightlist
\item
  Is the OR for ``female'' statistically significant, and what does this
  mean?
\end{enumerate}

Answer: The odds ratio for females was not statistically significant
from males. This means that the odds of female adolescents reporting
smoking is not significantly different from the odds of male adolescents
reporting smoking.

\begin{enumerate}
\def\labelenumi{\arabic{enumi})}
\setcounter{enumi}{5}
\tightlist
\item
  Model 2: Write a linear Random Coefficients model for the continuous
  outcome CESD, where adolescents (level 1) are nested in schools (level
  2). Include the following FE variables: female, parent education
  (reference level = college degree or more), and age\_w1. In addition,
  treat ``parent education'' as a random coefficient. Be sure to include
  all steps (micro, macro, Combined model) and to include any level 2
  and/or level 1 variance-covariance matrices. When writing the model
  include covariances in the variance-covariance matrix.
\end{enumerate}

\[ Depression_{ij} = \beta_{0j} + \beta_{1j}(lessthanhs_{ij}) + \beta_{2j}(hsgrad_{ij}) + \beta_{3j}(somecollege_{ij}) + \beta_{4j}(female_{ij}) + \beta_{5j}(age_{ij}) + \epsilon_{0ij}  \]

Macro 1:(Intercept) \[ \beta_{0j} = \beta_0 + \mu_{0j} \]

Macro 2a: (Slope for less than high school)
\[ \beta_{1j} = \beta_1 + \mu_{1j} \]

Macro 2b: (Slope for hs grad) \[ \beta_{2j} = \beta_2 + \mu_{2j} \]

Macro 2c: (Slope for some college) \[ \beta_{3j} = \beta_3 + \mu_{3j} \]

Combined:
\[ Depression_{ij} = \beta_{0} + \beta_{1}(lessthanhs_{ij}) + \beta_{2}(hsgrad_{ij}) + \beta_{3}(somecollege_{ij}) + \beta_{4}(female_{ij}) + \beta_{5}(age_{ij}) + \mu_{0j} + \mu_{1j}(lessthanhs_{ij}) + \mu_{2j}(hsgrad_{ij}) + \mu_{3j}(somecollege_{ij}) + \epsilon_{0ij} \]

\[ Level 2:
\left[\begin{array}{cc} 
\mu_{0j}\\
\mu_{1j}\\
\mu_{2j}\\
\mu_{3j}
\end{array}\right] 
\sim N  
\left(\begin{array}{cc}
0,
\left[\begin{array}{cc}
\sigma^2_{u0} &  \\ 
\sigma_{u0u1} & \sigma^2_{u1} &  \\
\sigma_{u0u2} & \sigma_{u1u2} & \sigma^2_{u2} &  \\
\sigma_{u0u3} & \sigma_{u1u3} & \sigma_{u2u3} & \sigma^2_{u3}
\end{array}\right]
\end{array}\right) \]

\[ Level 1: [\epsilon_{0ij}] \sim N(0,\sigma^2_{\epsilon0}) \]

Level 2 Variances: College degree or more
\[  Var(\mu_0) = \sigma^2_{u0}X^2_0  \]

Level 2 Variances: less than high school
\[  Var(\mu_0 + \mu_1) = \sigma^2_{u0}X^2_0 + 2\sigma_{u0u1}X_0X_1 + \sigma^2_{u1}X^2_1 \]

Level 2 Variances: high school degree
\[  Var(\mu_0 + \mu_2) = \sigma^2_{u0}X^2_0 + 2\sigma_{u0u2}X_0X_2 + \sigma^2_{u2}X^2_2  \]

Level 2 Variances: some college
\[  Var(\mu_0 + \mu_3) = \sigma^2_{u0}X^2_0 + 2\sigma_{u0u3}X_0X_3 + \sigma^2_{u3}X^2_3  \]

\begin{enumerate}
\def\labelenumi{\arabic{enumi})}
\setcounter{enumi}{6}
\tightlist
\item
  Using STATA, fit Model 2 but make the assumption that all covariances
  = 0. Then answer the following questions using the results.
\end{enumerate}

\begin{enumerate}
\def\labelenumi{\alph{enumi}.}
\tightlist
\item
  What is the between-school variance in the reference category (those
  with parent education = college degree or more)? Show any formulas you
  use.
\end{enumerate}

\begin{Shaded}
\begin{Highlighting}[]
\NormalTok{parent_dum <-}\StringTok{ }\KeywordTok{dummy.code}\NormalTok{(data}\OperatorTok{$}\NormalTok{parent_ed)}

\NormalTok{data <-}\StringTok{ }\KeywordTok{data.frame}\NormalTok{(data, parent_dum)}

\CommentTok{# colnames(data)}
\CommentTok{# str(data)}

\NormalTok{model2 <-}\StringTok{ }\KeywordTok{lmer}\NormalTok{(depression }\OperatorTok{~}\StringTok{ }\NormalTok{hs_grad }\OperatorTok{+}\StringTok{ }\NormalTok{less_than_hs }\OperatorTok{+}\StringTok{ }\NormalTok{some_college }\OperatorTok{+}
\StringTok{                 }\NormalTok{sex }\OperatorTok{+}\StringTok{ }\NormalTok{age }\OperatorTok{+}\StringTok{ }\NormalTok{(hs_grad }\OperatorTok{+}\StringTok{ }\NormalTok{less_than_hs }\OperatorTok{+}\StringTok{ }\NormalTok{some_college }\OperatorTok{||}\StringTok{ }\NormalTok{school),}
              \DataTypeTok{data =}\NormalTok{ data,}
              \DataTypeTok{REML =} \OtherTok{FALSE}\NormalTok{,}
              \DataTypeTok{control =} \KeywordTok{lmerControl}\NormalTok{(}\DataTypeTok{optimizer =} \StringTok{'Nelder_Mead'}\NormalTok{))}
\KeywordTok{summary}\NormalTok{(model2)}
\end{Highlighting}
\end{Shaded}

\begin{verbatim}
## Linear mixed model fit by maximum likelihood . t-tests use
##   Satterthwaite's method [lmerModLmerTest]
## Formula: depression ~ hs_grad + less_than_hs + some_college + sex + age +  
##     (hs_grad + less_than_hs + some_college || school)
##    Data: data
## Control: lmerControl(optimizer = "Nelder_Mead")
## 
##      AIC      BIC   logLik deviance df.resid 
##  42352.1  42426.5 -21165.0  42330.1     6395 
## 
## Scaled residuals: 
##     Min      1Q  Median      3Q     Max 
## -1.9965 -0.7034 -0.1872  0.5084  5.6774 
## 
## Random effects:
##  Groups   Name         Variance Std.Dev.
##  school   (Intercept)   0.55287 0.7436  
##  school.1 hs_grad       0.12640 0.3555  
##  school.2 less_than_hs  3.16423 1.7788  
##  school.3 some_college  0.07629 0.2762  
##  Residual              42.68472 6.5334  
## Number of obs: 6406, groups:  school, 132
## 
## Fixed effects:
##                Estimate Std. Error         df t value             Pr(>|t|)
## (Intercept)     3.33869    0.79982 1313.32004   4.174       0.000031857319
## hs_grad         1.54653    0.22042  219.25142   7.016       0.000000000028
## less_than_hs    2.66819    0.36807   79.77951   7.249       0.000000000238
## some_college    0.74319    0.21234  240.79061   3.500             0.000554
## sexfemale       1.44821    0.16465 6380.46053   8.796 < 0.0000000000000002
## age             0.43716    0.05071 1130.48262   8.620 < 0.0000000000000002
##                 
## (Intercept)  ***
## hs_grad      ***
## less_than_hs ***
## some_college ***
## sexfemale    ***
## age          ***
## ---
## Signif. codes:  0 '***' 0.001 '**' 0.01 '*' 0.05 '.' 0.1 ' ' 1
## 
## Correlation of Fixed Effects:
##             (Intr) hs_grd lss_t_ sm_cll sexfml
## hs_grad     -0.081                            
## less_thn_hs -0.031  0.270                     
## some_colleg -0.103  0.444  0.266              
## sexfemale   -0.150  0.002  0.005  0.016       
## age         -0.975 -0.038 -0.042 -0.018  0.044
\end{verbatim}

\begin{Shaded}
\begin{Highlighting}[]
\KeywordTok{as.data.frame}\NormalTok{(}\KeywordTok{VarCorr}\NormalTok{(model2))}
\end{Highlighting}
\end{Shaded}

\begin{verbatim}
##        grp         var1 var2        vcov     sdcor
## 1   school  (Intercept) <NA>  0.55286924 0.7435518
## 2 school.1      hs_grad <NA>  0.12640213 0.3555308
## 3 school.2 less_than_hs <NA>  3.16423238 1.7788289
## 4 school.3 some_college <NA>  0.07628729 0.2762015
## 5 Residual         <NA> <NA> 42.68472205 6.5333546
\end{verbatim}

Answer: The variation between schools for adolescents with parents that
have a college degree or more is .553

Level 2 Variances: College degree or more
\[  Var(\mu_0) = \sigma^2_{u0} = .553  \]

\begin{enumerate}
\def\labelenumi{\alph{enumi}.}
\setcounter{enumi}{1}
\tightlist
\item
  What is the between-school variance among those whose parents have
  education = some college? Show any formulas you use.
\end{enumerate}

Answer: The variation between schools for adolescents with parents that
have a some college is 1.237

\[  Var(\mu_0 + \mu_3) = \sigma^2_{u0} + 2\sigma_{u0u3} + \sigma^2_{u3}(9) = .553 + 0 + .076*9 = .553 + .684 = 1.237  \]

\begin{enumerate}
\def\labelenumi{\alph{enumi}.}
\setcounter{enumi}{2}
\tightlist
\item
  What is the between-school variance among those whose parents have
  education = high school degree? Show any formulas you use.
\end{enumerate}

Answer: The variation between schools for adolescents with parents that
have a high school degree is 13.209

Level 2 Variances: high school degree
\[  Var(\mu_0 + \mu_2) = \sigma^2_{u0} + 2\sigma_{u0u2} + \sigma^2_{u2}(4) = .553 + 0 + 12.656 = 13.209 \]

\begin{enumerate}
\def\labelenumi{\alph{enumi}.}
\setcounter{enumi}{3}
\tightlist
\item
  What is the between-school variance among those whose parents have
  education = less than high school? Show any formulas you use.
\end{enumerate}

Answer: The variation between schools for adolescents with parents that
have a less than a high school degree is .679

Level 2 Variances: less than high school
\[  Var(\mu_0 + \mu_1) = \sigma^2_{u0} + 2\sigma_{u0u1} + \sigma^2_{u1}(1) = .553 + 0 + .126 = .679 \]

\begin{enumerate}
\def\labelenumi{\alph{enumi}.}
\setcounter{enumi}{4}
\tightlist
\item
  Which group (those with parent education of college degree plus, some
  college, high school degree, or less than high school degree) has the
  largest betweenschool variance in CESD?
\end{enumerate}

Answer: The most variation in depression scores appears to be
adolescents with parents that have a high school degree.

\begin{enumerate}
\def\labelenumi{\arabic{enumi})}
\setcounter{enumi}{7}
\tightlist
\item
  On Canvas there is a file called ``Final Project Worksheet.'' Download
  this and fill it out to the best of your abilities. When you submit
  your homework, include a printed copy of this document (so I can
  provide some feedback) AND email me a copy of it. The document is in
  Word, so you should be able to fill it out electronically. I would
  like the electronic copy so I can begin to keep records about the
  final project you are proposing. We will schedule one-on-one meetings
  after this assignment is due, and I will use this worksheet to
  formally approve your final project proposal. If you already know you
  will not be doing a final project, there is no need to complete this
  worksheet. If you are not certain, it may make sense to complete it
  just in case so you can think through the possibility.
\end{enumerate}


\end{document}
